\documentclass{exam}
\usepackage{../../scripts/meta}

%%% Change these %%%%%%%%%%%%%%%%%%%%%%%%%%%%%%%%%%%%%%%%%%%%%%%%%%%%%%%%%%%%%%
\discnumber{0}
\title{}
\date{Graphs, Trees, Hypercubes}

%%%%%%%%%%%%%%%%%%%%%%%%%%%%%%%%%%%%%%%%%%%%%%%%%%%%%%%%%%%%%%%%%%%%%%%%%%%%%%%
\begin{document}
\maketitle
\rule{\textwidth}{0.15em}
\fontsize{12}{15}\selectfont
\thispagestyle{empty}


%%% Include topics here %%%%%%%%%%%%%%%%%%%%%%%%%%%%%%%%%%%%%%%%%%%%%%%%%%%%%%%
\section{General Comments - Graph Theory}
\begin{enumerate}
\item Make sure to take attendance!
\item Emphasize that the worksheet this week is long, it is not meant to be completely covered in 1.5 hours -- they should practice the problems that you did not get to.
\item Make sure that you are comfortable with the subtle differences between a walk and path or a tour and cycle. 
\item Especially emphasize 3-cycle and/or ham-path in Graph Theory.
\item If you are teaching in the beginning of the week, this will most likely be their first exposure to proving things about graphs. Go slow and make sure they understand why each step is necessary and how you would think of it
\item If low on time, skip Eulerian tours
\item Have students take special note of the 4 properties of trees we list. Some good general tree advice is to have those 4 written out exactly, which makes for a much easier time coming up with proofs
\item For Rooted Tree, a visual/intuitive explanation rather than a formal proof is fine. 
\item If there is time, try and emphasize the Spanning Tree problem.
\item Hypercubes; M/T/W: just talk them through the definition and ask if they need anything clarified from lecture. Take a look at the lecture notes before teaching to see what they covered
\item Remember that there are two major definitions of a hypercube. 
\item TH/F: they will probably be more comfortable with hypercubes. Do these 2 questions as time permits.
\item The 2 questions are nice since you can use the recursive definition to make one pretty easy and the binary definition to do the other one, driving home this point of both representations being important. You can also use a visual definition if you feel that it can convey the intuition better.
\end{enumerate}

\section{General Comments - Modular Arithmetic}
\begin{enumerate}
\item FLT Proof is important to go through; make sure they understand it. The key step in the proof is using the bijection in proving the equivilance of two sets.
\end{enumerate}


\clearpage 

%%% Include topics here %%%%%%%%%%%%%%%%%%%%%%%%%%%%%%%%%%%%%%%%%%%%%%%%%%%%%%%
\section{Questions}
\subsection{Graph Theory}
\begin{enumerate}
\subimport{../../topics/graphs/meta/}{3_cycle.tex}
% \subimport{../../topics/graphs/meta/}{build_up_error.tex}
\end{enumerate}

\subsection{Hypercubes}
\begin{enumerate}
\subimport{../../topics/hypercubes/meta/}{introduction.tex}
\end{enumerate}
%%%%%%%%%%%%%%%%%%%%%%%%%%%%%%%%%%%%%%%%%%%%%%%%%%%%%%%%%%%%%%%%%%%%%%%%%%%%%%%


\end{document}
