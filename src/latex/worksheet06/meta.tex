Week Specific Meta:

Mandatory Questions:
Berlekamp-Welsh: 
Make sure students understand why the algorithm works
Countability: 
#1-3; 
The set question
If A and B are both countable, then AxB is countable. (True/False)
Self Reference: 
	Halting Problem Proof
	All 3 exercises (for Mon-Wed people); As needed (for Thur-Fri people)
Intro to Counting: 
	Mon-Wed people will not get to this
	Poker
	Solving Equations

Topic/Questions Specific Notes:

Berklekamp-Welsh:

Countability:
Make sure that students are familiar with the common sets (integers, rationals, etc.)
The Hotel Argument 
https://www.youtube.com/watch?v=6NlwpEArfwk&list=PL-XXv-cvA_iD8wQm8U0gG_Z1uHjImKXFy&index=13 
this provides a good mechanism for explaining the intuition behind finding a bijection
depending on the level of preparedness of your students you may or may not have to draw out the “hotels” on the board
This is a good question to determine if they go to lecture
Cantor-Bernstein Theorem:
This theorem is not focused on (maybe not even really mentioned) in lecture, so the name and formal statement is just a neat thing to mention, BUT the technique that comes from it, namely the technique used to prove Qs countability, is very useful. Said technique states that if |A| ≤ |B| and |B| ≤ |A| then |A| = |B|.
You can do a proof of why Q is countable here.
We know that |N| ≤ |Q| because every natural number is a rational number.
Just need to show that |Q| ≤ |N|.
Feel free to do a short version of the spiral proof. They will probably go over this in lecture, but if they didn’t get it there you should cover it here. (Ask!)

Self-Reference:
Turing program question is typically included in notes, but can be confusing. Definitely want to make sure students understand why this proof makes sense. Second part of that first question is good to understand but not central to the concepts of self-reference/computability. For question 2, suggest that students try to use the program to solve the halting problem (which if it can proves that the program cannot exist.

Intro to Counting:
Definitely make sure to lecture on the rules of counting. Rather than just stating them, it will probably be better to do the examples with them in mind – it’s a lot easier to understand them when you get why they’re used
Make sure to do the first one, as it emphasizes intuition of permutations vs. combinations, etc., and shouldn’t take too long
Do the easier Starbucks counting ones to give them practice
Solving equations-like stuff comes up decently often, so I definitely recommend going through that

