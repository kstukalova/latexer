Important Questions
1.1 - 1
1.1 - 3,5 are esp. Good
#5 on pg. 9
4 in continuous distributions

General Guidelines
They should be decently familiar with Markov/Chebyshev’s Inequalities – so make sure to focus on those, and that they are comfortable with them moving on. They are tested quite often
I think it’s good to present confidence intervals as an application of the inequalities, which should make it more intuitive. Up to you, whatever makes most sense to you and your students
Conditional Expectation is difficult. This will be covered in lecture Friday, so students will not have seen it, or be very confused about it. Make sure to prepare an explanation
Draw a picture to show that conditioning creates a new random variable with a new distribution. Below, Figure 9 of note 26 does so by defining a random variable E[Y | X], which is a function of X. E[Y | X] is a random variable because giving E[Y | X] an outcome makes X return a number, from which E[Y | X] (which is a number) can be calculated. 
I.e. E[Y | X] is a composition of a function on X and a function on the outcome space.
I.e. E[Y | X] : Ω → R
% LEAVE OPEN THERE IS A PICTURE
Continuous has the lowest priority due to it being covered (most likely) after Thanksgiving
Make sure they understand why we integrate/derive when to manipulate between CDF/PDF
Just like the intro to RV’s recently, have them be comfortable with defining continuous RV’s
Move onto continuous distributions only if you have time. I think it’s intuitive to present them as the continuous analogs to their discrete counterparts – but again, do whatever works best
Make sure they understand why we use the values that we use for PDF’s/CDF’s
A lot of the problems here are just plug and chug, so shouldn’t be too difficult

Overleaf: https://www.overleaf.com/6811113rjngpbjrrykq#/23200247/
